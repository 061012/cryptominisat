%\documentclass[runningheads]{llncs}
\documentclass[final]{ieee}

\usepackage{microtype} %This gives MUCH better PDF results!
%\usepackage[active]{srcltx} %DVI search
\usepackage[cmex10]{amsmath}
\usepackage{amssymb}
\usepackage{fnbreak} %warn for split footnotes
\usepackage{url}
%\usepackage{qtree} %for drawing trees
%\usepackage{fancybox} % if we need rounded corners
%\usepackage{pict2e} % large circles can be drawn
%\usepackage{courier} %for using courier in texttt{}
%\usepackage{nth} %allows to \nth{4} to make 1st 2nd, etc.
%\usepackage{subfigure} %allows to have side-by-side figures
%\usepackage{booktabs} %nice tables
%\usepackage{multirow} %allow multiple cells with rows in tabular
\usepackage[utf8]{inputenc} % allows to write Faugere correctly
\usepackage[bookmarks=true, citecolor=black, linkcolor=black, colorlinks=true]{hyperref}
\hypersetup{
pdfauthor = {Mate Soos},
pdftitle = {CryptoMiniSat v4.4},
pdfsubject = {SAT Race 2015},
pdfkeywords = {SAT Solver, DPLL},
pdfcreator = {PdfLaTeX with hyperref package},
pdfproducer = {PdfLaTex}}
%\usepackage{butterma}

%\usepackage{pstricks}
\usepackage{graphicx,epsfig,xcolor}

\begin{document}
\title{The CryptoMiniSat-4.4 set of solvers at SAT Race 2015}
\author{Mate Soos}

\maketitle
\thispagestyle{empty}
\pagestyle{empty}

\section{Introduction}
This paper presents the conflict-driven clause-learning SAT solver CryptoMiniSat v4.4 (\emph{CMS4.4}) as submitted to SAT Race 15. CMS4.4 aims to be a modern, open-source SAT solver that allows for multi-threaded in-processing techniques while still retaining a strong CDCL component. In this description only the features relative to CMS4.4, the previous year's submission, are explained. Please refer to the previous years' description for details. In general, CMS4.4 is a in-processing SAT solver that usues optimized datastructures and finely-tuned timeouts to have good control over both memory and time usage of simplification steps.

\subsection{Using watchlists as occurrence lists}
As per lingeling \cite{lingeling}, CMS4.4 now uses the watchlist to store occurrence lists (when they are needed) and related occurrence information such as data related to looking for XOR clauses or gates. This significantly reduces the memory overhead and, due to cache locality, increases speed.

\subsection{Removal of uneeded code}
Over the years, many lines of code has been added to CMS that in the end didn't help and often was detrimental to both maintinability and efficiency of the solver. Many such additions have now been removed. This simplifies understanding and developing the system. Further, it allows the system to be more lean especially in the tight loops such as propagation and conflict analysis where most of the time is spent.

\subsection{Integration of ideas from SWDiA5BY A26}
Some of the ideas from SWDiA5BY A26\cite{swdia} have been included into CMS. In particular, the clause cleaning system employed and the switching restart have both made their way into CMS. Further, SWDiA5BY A26 was used as a test-bed against CMS to clean up the codebase from unwated and unneeded elements.

\subsection{Incremental solving}
Incremental solving for a in-processing solver is not trivial and many bugs have been found in fuzzing the incremental solving interface. The fuzzer developed for this purpose contains more than 1000 lines of python and allows for testing both the incremental and the DRAT~\cite{drat} interface of the solver.

\subsection{Versions}
Two versions of the solver were submitted to the race. The version "4.4" has been tuned to use a specific set of constants using a cluster and machine learning methods. The second version, "4.4-auto" runs for approx 100K conflicts and then based on static data extracted from the simplified CNF chooses one of 7 solving strategies. Both variants have been tuned on past competitions' problem instances.

\section*{Acknowledgements}
The author would like to thank Marius Lindauer for his help in tuning the solver using the SpySMAC suite and the Postdam university computing cluster.

\bibliographystyle{splncs}
\bibliography{sigproc}

\vfill
\pagebreak

\end{document}
